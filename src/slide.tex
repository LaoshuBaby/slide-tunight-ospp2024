\documentclass{beamer}
\usepackage{ctex, hyperref}
\usepackage[T1]{fontenc}

% other packages
\usepackage{latexsym,amsmath,xcolor,multicol,booktabs,calligra}
\usepackage{graphicx,pstricks,listings,stackengine}

\author{快乐的老鼠宝宝}
\title{基于前端技术栈的\\ OpenStreetMap 中公共交通关系编辑器}
\subtitle{Frontend technology based OpenStreetMap PT relation editor}
\institute{OSMChina}
\date{2024年5月11日}
\usepackage{Tsinghua}

% defs
\def\cmd#1{\texttt{\color{red}\footnotesize $\backslash$#1}}
\def\env#1{\texttt{\color{blue}\footnotesize #1}}
\definecolor{deepblue}{rgb}{0,0,0.5}
\definecolor{deepred}{rgb}{0.6,0,0}
\definecolor{deepgreen}{rgb}{0,0.5,0}
\definecolor{halfgray}{gray}{0.55}

\lstset{
    basicstyle=\ttfamily\small,
    keywordstyle=\bfseries\color{deepblue},
    emphstyle=\ttfamily\color{deepred},    % Custom highlighting style
    stringstyle=\color{deepgreen},
    numbers=left,
    numberstyle=\small\color{halfgray},
    rulesepcolor=\color{red!20!green!20!blue!20},
    frame=shadowbox,
}


\begin{document}

\kaishu
\begin{frame}
    \titlepage
    \begin{figure}[htpb]
        \begin{center}
            \includegraphics[width=0.2\linewidth]{figure/tuna.pdf}
        \end{center}
    \end{figure}
\end{frame}

\begin{frame}
    \tableofcontents[sectionstyle=show,subsectionstyle=show/shaded/hide,subsubsectionstyle=show/shaded/hide]
\end{frame}

\section{项目基础信息}

\begin{frame}
    \begin{itemize}
        \item 项目难度: 进阶 / Medium
        \item 项目社区导师:快乐的老鼠宝宝 / LaoshuBaby
        \item 导师联系方式:\href{mailto:me@jacksonzhao.email}{me@jacksonzhao.email}
    \end{itemize}
\end{frame}

\section{项目描述}

\begin{frame}{高情商:具有较多技术债的现有项目}
    \quad \quad OpenStreetMap是英国人Steve Coast于2004年发起的以知识开放为原则的地图项目。\\
    \quad \quad 其数据模型对于公共交通等现实中抽象要素采用的关系描述方式较为复杂,OSM基金会从2016年至今每年都往GSOC提交公共交通相关的项目,但是成果和具有技术债的工具链关系密切,可用性不强。\\
    \quad \quad 本项目旨在以现代化的前端技术栈,实现一个可跨平台使用且操作简单交互明了的编辑器。\\
    \quad \quad 本项目过程中在交互和功能性上会与OSM中国社区合作。
\end{frame}

\begin{frame}
    \quad \quad OpenStreetMap is a map project initiated in 2004 by Steve Coast of the UK, based on the principle of knowledge openness. 
    Its data model employs a complex method of describing relationships for abstract elements in reality, such as public transport. \\
    \quad \quad Since 2016, the OSM Foundation has submitted projects related to public transport to GSOC annually, 
    but the outcomes are closely tied to a technically indebted toolchain, resulting in poor usability. \\
    \quad \quad This project aims to develop a cross-platform editor using a modern front-end technology stack, which is easy to use with clear interactions. \\
    \quad \quad During this project, there will be collaboration with the OSMChina community on interaction and functionality.
\end{frame}

\section{项目产出要求}

\begin{frame}
    \begin{enumerate}
        \item 能够在编辑器内添加站点并创建线路关系(关系内成员顺序可变)或删除关系并通过OSM API 0.6上传。
        \item 能在只有站点数据的情况下,根据图幅中路网和控制点自动计算可行的经由,和基于前述经由自动切割较长的路径以便于添加关系。
        \item 需实现至少包含主要功能的原型。
    \end{enumerate}
\end{frame}

\begin{frame}
    \begin{enumerate}
        \item Ability to add stations and create or delete route relationships (with variable member order) within the editor and upload them via OSM API 0.6.
        \item Ability to automatically calculate viable routes based on the road network and control points in the map, and automatically split longer paths for easier relationship addition, when only station data is available.
        \item Implementation of at least a prototype containing the main functionalities.
    \end{enumerate}
\end{frame}

\section{项目技术要求}

\begin{frame}
    \begin{enumerate}
        \item 能够使用前端技术栈开发具有复杂界面的跨平台程序,对WebGL有了解。(有用过Cesium.js/MapboxGL等WebGL的GIS框架属于加分项)
        \footnote{(可能还需要一定的图形学和WASM知识,视最终技术栈选择而定。)}
        \item 有一定算法能力,对导航算法的实现有一定的了解。(有用过开源的导航路由工具如OSRM、Graphhopper、Valhalla等属于加分项。)
        \item 对OpenStreetMap的数据模型和API有所了解。(对社区文化和协作方式有了解或曾使用过相关数据进行分析等利用可视为加分项)
    \end{enumerate}
\end{frame}

\begin{frame}
    \begin{enumerate}
        \item Ability to use front-end technology stack to develop complex interface cross-platform programs, with knowledge of WebGL. (Experience with WebGL GIS frameworks such as Cesium.js/MapboxGL is a plus)
        \footnote{(Depending on the final choice of technology stack, knowledge of graphics and WASM might be required.)})
        \item Adequate algorithmic skills, with some understanding of navigation algorithm implementation. (Experience with open-source navigation routing tools like OSRM, Graphhopper, Valhalla, etc., is a plus.)
        \item Understanding of the OpenStreetMap data model and API. (Knowledge of community culture and collaboration methods, or previous use of related data for analysis or other purposes, is very important.)
    \end{enumerate}
\end{frame}


\section{相关的开源软件仓库列表}

\begin{frame}
    \begin{enumerate}
        \item \url{https://github.com/openstreetmap/iD} \\
        (关系编辑能力不强)
        \item \url{https://github.com/JOSM/pt_assistant} \\
        (不能独立于JOSM使用,较笨重)
        \item \url{https://github.com/Zaczero/osm-relatify} \\
        (一个偏实验性的编辑器原型,不适合大量编辑)
    \end{enumerate}
    \begin{figure}[htpb]
        \centering
        \includegraphics[width=0.2\linewidth]{figure/tuna.pdf}
        \caption{这个校徽就是矢量图}
    \end{figure}
\end{frame}

\section{吐槽}

\begin{frame}
    我一个后端佬,怎么就来带前端项目了呢?所以说一个人的定位,不仅要看,也要……
\end{frame}


\section{计划进度}
\begin{frame}
    \begin{itemize}
        \item 一月:完成文献调研
        \item 二月:复现并评测各种Beamer主题美观程度
        \item 三、四月:美化THU Beamer主题
        \item 五月:论文撰写
    \end{itemize}
    \begin{table}[h]
        \centering
        \begin{tabular}{c|c}
            Microsoft\textsuperscript{\textregistered}  Word & \LaTeX \\
            \hline
            文字处理工具 & 专业排版软件 \\
            容易上手,简单直观 & 容易上手 \\
            所见即所得 & 所见即所想,所想即所得 \\
            高级功能不易掌握 & 进阶难,但一般用不到 \\
            处理长文档需要丰富经验 & 和短文档处理基本无异 \\
            花费大量时间调格式 & 无需担心格式,专心作者内容 \\
            公式排版差强人意 & 尤其擅长公式排版 \\
            二进制格式,兼容性差 & 文本文件,易读、稳定 \\
            付费商业许可 & 自由免费使用 \\
        \end{tabular}
    \end{table}
\end{frame}


\section{参考文献}

\begin{frame}[allowframebreaks]
    \bibliography{ref}
    \bibliographystyle{alpha}
    % 如果参考文献太多的话,可以像下面这样调整字体:
    % \tiny\bibliographystyle{alpha}
\end{frame}

\begin{frame}
    \begin{center}
        {\Huge\calligra Thanks!}
    \end{center}
\end{frame}

\end{document}